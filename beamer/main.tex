\documentclass[20pt, fontset=none]{beamer}

% Set the same paper width and height as .pptx(default is [4:3]).
% [4:3] width = 25.4cm height = 19.05cm
% [16:9] width = 25.4cm height = 14.288cm
\usepackage{geometry}
% \geometry{paperwidth = 25.4cm, paperheight = 19.05cm}
\geometry{paperwidth = 25.4cm, paperheight = 14.288cm}
\setbeamersize{text margin left = 2cm}
\setbeamersize{text margin right = 2cm}

% text-align justify
\usepackage{xpatch}
\usepackage{ragged2e}
\justifying\let\raggedright\justifying
% itemize text-align justify
% \xpatchcmd{\itemize}{\raggedright}{\justifying}

% Use custom theme `NESC`
\usepackage{theme/beamerfontthemenesc}
\usepackage{theme/beamercolorthemenesc}
\usepackage{theme/beamerinnerthemenesc}
\usepackage{theme/beamerouterthemenesc}
\usepackage{theme/beamerthemenesc}

% % Basic info on title page
\conference{Group Meeting}
\title{An Introduction to something or something}
\subtitle{how to ...}
\reporter{Shiki}
\author{Shiki \inst{1}}
\institute{\inst{1} where are you from}
\email{xxx@xx.com}
\date{\today}

% Add table of contents at the begin of section.
\AtBeginSection[]{
  \begin{frame}[plain]
    \tableofcontents[
      sectionstyle=show/shaded,
      subsectionstyle=show,
      hideothersubsections]
    \addtocounter{framenumber}{-1}
  \end{frame}
}

% \usepackage[backend=bibtex, sorting=none]{biblatex}
% \addbibresource{main.bib}
% \defaultfontfeatures[FandolSong]{Script=Default}
% \usepackage{fontspec}
% \usepackage{xeCJK}

\usepackage{amsmath, amssymb, amsfonts}
% \usepackage{color}
\usepackage{graphicx,hyperref,url}
% \usepackage{comment}
\usepackage{lmodern}
\usepackage{xcolor}
% \usepackage{subcaption}
% \usepackage{booktabs}
\usepackage{bookmark}
% \usepackage{graphbox}
% \usepackage{arydshln}
\usepackage{blindtext}
\parindent=0pt
% let math use serif font
\usefonttheme[onlymath]{serif}


\begin{document}
  \begin{sloppypar}
  \include{content/titlepage}
  \include{content/toc}
  \include{content/item}
  % !TEX root = ../main.tex

\section{Block Env}

\begin{frame}
  \begin{block}{standard block}
    Content
  \end{block}
  \begin{alertblock}{alert block}
    Content
  \end{alertblock}
  \begin{exampleblock}{example block}
    content
    Content \cite{scharnowskifuzzware}
  \end{exampleblock}
\end{frame}

\begin{frame}
  \begin{theorem}
    Content
  \end{theorem}
  \begin{definition}
    Content
  \end{definition}
  \begin{proof}
    Content
  \end{proof}
  \begin{lemma}
    Content
  \end{lemma}
\end{frame}
\begin{frame}
  \begin{corollary}
    Content
  \end{corollary}
  \begin{example}
    \begin{equation}
      \alpha = 14
    \end{equation}
    Content
  \end{example}
\end{frame}
  % !TEX root = ../main.tex

\section{Code Env}

\begin{frame}[fragile]
  \begin{commentBox}{My first comment}{first}
    My first comment here! \faTree
  \end{commentBox}

  \begin{codeBox}{python}{test.py}{{2-4}}{test1}
    import numpy as np
    a = 1
    b = 2
    print("{}".format{a+b})
  \end{codeBox}

\end{frame}

\begin{frame}
  \codeFileBox{python}{code/test.py}{test.py}{3}{7}{{3}}{test2}
  See \fullref{comm:first} for details. Two programs are listed: \Fullref{program:test1}, 
  \fullref{program:test2}
\end{frame}

  % !TEX root = ../main.tex

\section{Reference}

% reference slide.
\begin{frame}[allowframebreaks]
  \nocite{*}
  \bibliographystyle{ieeetr}
  \bibliography{main}
\end{frame}

  \end{sloppypar}
\end{document}
